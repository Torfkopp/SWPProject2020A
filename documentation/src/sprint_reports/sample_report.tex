%! Install TeXiFy IDEA plugin if you want to be able to properly edit .tex files in IntelliJ
%! (has code completion, etc.)

\documentclass[12pt,a4paper, oneside]{article}
\usepackage[utf8]{inputenc}
\usepackage[T1]{fontenc}
\usepackage[english,german]{babel}
\usepackage[style=german]{csquotes}
\usepackage{graphicx}

\author{Uni Oldenburg, SWP2020 Gruppe A}

\begin{document}

    \begin{titlepage}
        \pagestyle{empty}
        \begin{center}

            \begin{figure}[h]
                \centering
                \includegraphics[width=0.35\textwidth]{img/Logo.jpg}
            \end{figure}

            \bigskip \bigskip \noindent
            \textsc{\textbf{\LARGE Softwareprojekt:}} \par \bigskip \noindent
            \textsc{\textbf{\LARGE Projekttagebuch}}


            \par \bigskip \bigskip \bigskip \bigskip \bigskip \noindent
            {\Large Gruppe A} \par \medskip \noindent

            \par \bigskip \bigskip \bigskip \bigskip \bigskip \bigskip \noindent
            \textit{\Large Wintersemester 2020/21 und} \par \noindent
            \textit{\Large Sommersemester 2021}

            \par \bigskip \bigskip \bigskip \bigskip \bigskip \bigskip \noindent
            \par \bigskip \bigskip \bigskip \noindent
            {\Large Sprintanalyse} \par \medskip \noindent

        \end{center}
    \end{titlepage}

    \tableofcontents
    \pagebreak


    \section{Sprinttagebuch: Sprint-Nr.}
    \underline{Name des Sprints:}
    \\
%Hier kommt der Name des Sprints

    \noindent
    \\
    \underline{Zeitraum des Sprints:}
    \\
%Hier kommt der Zeitraum

    \noindent
    \\
    \underline{Ziel des Sprints:}
    \\
%Hier kommt das Ziel

    \noindent
    \\
    \underline {Team:}
    \\
    Sven Ahrens, Alwin Bossert, Aldin Dervisi, Marvin Drees, Mario Fokken,
    Timo Gerken, Finn Haase, Temmo Junkhoff, Maximilian Lindner, Steven Luong,
    Phillip-André Suhr, Eric Vuong


    \section{Vorgänge}
%Aufzeigen bzw. Übersicht der einzelnen Vorgänge in diesem Sprint

    \subsection{Sprinterfolg}
% Hier kann beispielsweise ein Bild des Burndown-Diagramms des jeweiligen Sprints hinzugefügt werden und als Fließtext eine Erklärung dazu.

    \subsection{Sprintprobleme bzw. Hindernisse}


    \section{Erkenntnis aus der Retroperspektive}


    \section{sonstige Anmerkungen}
%Falls es keine sonstigen Anmerkungen gibt, dann kann diese Sektion rausgelöscht werden.


    \section{Fazit}

\end{document}
