\documentclass[12pt,a4paper, oneside]{article}
\usepackage[utf8]{inputenc}
\usepackage[T1]{fontenc}
\usepackage[english,german]{babel}
\usepackage[style=german]{csquotes}
\usepackage{graphicx}

\author{Uni Oldenburg, SWP2020 Gruppe A}

\begin{document}

    \begin{titlepage}
        \pagestyle{empty}
        \begin{center}

            \begin{figure}[h]
                \centering
                \includegraphics[width=0.35\textwidth]{Logo.jpg}
            \end{figure}

            \bigskip \bigskip \noindent
            \textsc{\textbf{\LARGE Softwareprojekt:}} \par \bigskip \noindent
            \textsc{\textbf{\LARGE Projekttagebuch}}


            \par \bigskip \bigskip \bigskip \bigskip \bigskip \noindent
            {\Large Gruppe A} \par \medskip \noindent

            \par \bigskip \bigskip \bigskip \bigskip \bigskip \bigskip \noindent
            \textit{\Large Wintersemester 2020/21 und} \par \noindent
            \textit{\Large Sommersemester 2021}

            \par \bigskip \bigskip \bigskip \bigskip \bigskip \bigskip \noindent
            \par \bigskip \bigskip \bigskip \noindent
            {\Large Sprintanalyse} \par \medskip \noindent

        \end{center}
    \end{titlepage}

    \tableofcontents
    \pagebreak



    \section{Sprinttagebuch: Sprint-Nr. 9}
    \underline{Name des Sprints:}
    \\Sprint-Nr. 9: Fear of the Dark
%Fear of the Dark

    \noindent
    \\
    \underline{Zeitraum des Sprints:}
    \\15. April 2021 - 04. Mai 2021
%15. April 2021 - 04. Mai 2021

    \noindent
    \\
    \underline{Ziel des Sprints:}
    \\UX und Feinschliff

%Hier kommt das Ziel

    \noindent
    \\
    \underline {Team:}
    \\
    Sven Ahrens, Alwin Bossert, Aldin Dervisi, Marvin Drees, Mario Fokken,
    Timo Gerken, Finn Haase, Temmo Junkhoff, Maximilian Lindner, Steven Luong, Phillip-André Suhr, Eric Vuong


    \section{Vorgänge}
%Aufzeigen bzw. Übersicht der einzelnen Vorgänge in diesem Sprint
    \begin{itemize}
        \item SWP2020A-131: ANmi eine Hilfsanzeige für meine aktuellen Möglichkeiten haben (3 Story Points)

        \item SWP2020A-132: Spieler soll eine Regelübersicht einsehen können (1 Story Point)

        \item SWP2020A-140: ANmi, dass die Längste Handelsstraße und Größte Rittermacht automatisch korrekt zugewiesen werden (2  Story Points)

        \item SWP2020A-197: ANmi die optionale Gründerphase zu Partiebeginn nutzen können (3 Story Points)

        \item SWP2020A-204: ANmi, dass es eine maximale Zugzeit gibt, nach Ablauf derer End Turn vom System forciert wird (3 Story Points)

        \item SWP2020A-208 *: ANmi im Lobbychat sehen können, wer der Lobby beigetreten bzw. sie verlassen hat (1 Story Point)

        \item SWP2020A-228: ANmi eine Lobby mit einem Passwort schützen können, damit keine Fremden beitreten können (1 Story Point)

        \item SWP2020A-230: ANmi sehen können, wie lange die aktuelle Partie bereits gespielt wird (1 Story Point)

        \item SWP2020A-234: ANmi beim Bauen sehen können, an welchen Stellen ich (noch) bauen darf (2 Story Points)

        \item SWP2020A-245: ANmi als Owner einer Lobby den Ownerstatus auf einen bestimmten Mitspieler übertragen können (1  Story Point)

        \item SWP2020A-246:	ANmi vor Start einer Partie durch Zufall die Zugreihenfolge bestimmen können (1 Story Point)

        \item SWP2020A-255:	Bericht für Sprint 08 erstellen (2 Story Points)

        \item SWP2020A-261:	ANmi in einer Partie sehen können, in welcher Runde sich das Spiel befindet (1 Story Point)

        \item SWP2020A-266:	TradeWith*Presenter abstrahieren (1 Story Point)

        \item SWP2020A-267:	Nach Restart Lobby werden bei anderen Teilnehmern noch die Ressourcenanzahlen in der Memberliste und der Systemnachrichtenverlauf im Chat angezeigt (2 Story Points)

        \item SWP2020A-268:	"Change Move Time"-Button und -Textfeld werden bei nicht-Ownern angezeigt (1 Story Point)

        \item SWP2020A-269:	"Confirm New Password"-Feld in ChangeAccountDetails einfügen (0 Story Points)

        \item SWP2020A-270:	Spielerzahleinstellung bei Lobbyerstellung rausnehmen und Standardwert auf 3 Spieler setzen (1 Story Point)

        \item SWP2020A-271:	AEmi das Loglevel in einer Configoption festlegen können (1 Story Point)

        \item SWP2020A-274:	Im Gameservice fehlen bei vielen Methoden Checks, ob die entsprechende Aktion überhaupt ausgeführt werden darf (2 Story Points)

        \item SWP2020A-277:	ANmi eine Option haben, damit automatisch gewürfelt wird (1 Story Point)

        \item SWP2020A-278:	AEmi Beschränkungen auf Nutzernamen und Lobbynamen haben (1 Story Point)

        \item SWP2020A-279:	Lobbybeitritt ist bereits auf dem XYZ has won"-Screen möglich (1 Story Point)

        \item SWP2020A-281:	Redundantes setOnCloseRequest und window-Attribut im MainMenuPresenter entfernen (1 Story Point)

        \item SWP2020A-282:	Bereits vor dem Würfeln kann mit durch Anklicken eines Users mit ihm Resourcen tauschen (1 Story Point)

        \item SWP2020A-283:	Im serverseitigen Userservice muss die IllegalArgumentException vernünftig abgefangen werden (1 Story Point)

        \item SWP2020A-284:	UserServiceTest.registerSecondUserWithSameName (Server) ist erfolgreich, obwohl er korrekterweise eine Exception abfangen müsste (1 Story Point)

        \item SWP2020A-285:	Accountlöschung sollte eine Bestätigung erfordern (1 Story Point)

        \item SWP2020A-286:	ANmi die Straßenbaukarte mit ihrem Effekt nutzen können (2 Story Points)

        \item SWP2020A-288:	Passwort sollte bereits auf Clientseite gehasht werden, sodass es nur noch im Textfeld unverschlüsselt vorliegt (1 Story Point)

        \item SWP2020A-290:	ANmi über eine Configoption ein Theme auswählen können (3 Story Points)

        \item SWP2020A-295:	Turnindicator sollte Spielernamen in dessen Farbe anzeigen (1 Story Point)

        \item SWP2020A-296 *: SessionManagementException im AuthenticationService wird nicht behandelt (0 Story Points)

        \item SWP2020A-297 *: "java.io.IOException: Eine vorhandene Verbindung wurde vom Remotehost geschlossen" bei Serverabbruch braucht i18n (0 Story Points)

        \item SWP2020A-298 *: Die AbstractX zwischen Interfaces und Implementierungen können entfernt werden (0 Story Points)

        \item SWP2020A-300 *: Spielrelevante Buttons werden erst freigeschaltet, wenn alle Spieler mit dem Steuerbezahlen fertig sind (2 Story Points)

        \item SWP2020A-301 *: SessionManagement kann nie die Session eines eingeloggten Nutzers finden und macht damit u.A. Handel kaputt  (1 Story Point)

        \item SWP2020A-304 *: LOG.isDebugEnabled() entfernen und LOG.debug-Formatting ausnutzen (0 Story Points)

        \item SWP2020A-305 *: Bei Clientwechsel übernimmt die "Automatic Roll"-Checkbox nicht den vorherigen Zustand (0 Story Points)

        \item SWP2020A-309 *: Beim Spielen einer Ritterkarte werden die spielrelevanten Buttons nicht disabled während der Räuber noch nicht neu platziert ist (1 Story Point)

        \item SWP2020A-310 *: Monopolkarte nimmt anderen Spielern nicht alle Ressourcen der gewählten Sorte weg, sondern nur 1 und aktualisiert nicht die Inventare der bestohlenen Nutzer (0 Story Points)

        \item SWP2020A-311 *: H2BasedUserStoreTest.getAllUsers schlägt in IntelliJ fehl (UserServiceTest sollte ein @AfterAll bekommen, das den dort verwendeten Nutzer entfernt) (0 Story Points)
    \end{itemize}

    \textbf{Nicht abgeschlossene Vorgänge:}
    \begin{itemize}
        \item SWP2020A-223:	Diverse Klassen in Common aufteilen, um Zugriffsrechte einzuschränken und Enums verstärkt zu verwenden (3 → 5 Story Points)
    \end{itemize}

    \textbf{Aus dem Sprint entfernte Vorgänge:}
    \begin{itemize}
        \item SWP2020A-276:	Weitere Dokumentation und Verbesserung vom CommandService (3 Story Points)
    \end{itemize}


    * = Nachgezogene Task/User Story




    \subsection{Sprinterfolg}
% Hier kann beispielsweise ein Bild des Burndown-Diagramms des jeweiligen Sprints hinzugefügt werden und als Fließtext eine Erklärung dazu.
    \includegraphics[width=1\textwidth]{burndown 9.PNG}
    Dieser Sprint startete mit 50 Story-Points, also ungefähr so viel wie in den letzten Sprints. Jedoch haben wir im Laufe des Sprints oder erst nach dem Sprint bemerkt, dass wir einige Tasks unterschätzt haben, wie zum Beispiel die SWP2020A-223, welche aus sicherheitstechnischen Gründen Klassen in Common aufteilen sollte. Am Ende des Sprints sind nur 5 Story Points übrig geblieben, trotz dass wir mit 50 gestartet sind und vieles noch nachgezogen haben.


    \subsection{Sprintprobleme bzw. Hindernisse}
    Probleme waren vermutlich, dass wir den selben Sprintumfang aus den letzten Sprints erzielen wollten, jedoch durch Semesterbeginn nicht mehr so viel Flexibilität fürs SWP hatten. Dazu wurden Tasks nicht sehr effizient verteilt, da einige, auch wenn sie früher angefangen haben, schon in der Mitte des Sprints mit ihren Tasks fertig waren, während andere noch keine Task fertiggestellt haben. Dazu haben sich auch Reviews in die Länge gezogen, was man aus dem Burndown ein wenig entnehmen kann.

    \section{Erkenntnis aus der Retrospektive}
    Start:
    \begin{itemize}
        \item Entwurf/Ansatz nach einer Woche bei 3+ Story Points Task
        \item Bei umfangreichen Stories: Pair-Programming
        \item Recherche bei Verständnisproblemen
    \end{itemize}
    Stop:
    \begin{itemize}
        \item Bei den Reviews ist früher kommen besser
        \item Kummerkasten aktiv nutzen, damit wir 2MB Ram Jira Tools nicht mehr nutzen müssen
    \end{itemize}
    Weiter so:
    \begin{itemize}
        \item Solo-Programming bei kleinen Stories
        \item (Discord) Ansprechbarkeit
        \item Gründliche Reviews
    \end{itemize}

    \section{Fazit}
    Obwohl es für viele anfangs nicht voranging, haben wir uns am Ende des Sprints wieder gefunden und mit Ausnahme einer komplexen, unterschätzten Task, alles fertigstellen können. Trotzdessen zieht sich das Burndown und die Retrospektive wieder durch das alte Muster, das wir nach der Hälfte des Sprints quasi keinen Fortschritt haben und der Guideline auch nie wirklich in die Nähe kommen können. Trotzdem konnten wir fast alle Tasks gründlich und sauber lösen, um das Ziel, die User Experience zu verbessern, zu erfüllen. Mit dem Sprint haben wir es als Team geschafft, dem User mehr Konfigurationsmöglichkeiten über seine Einstellungen zu bieten, seien es In-Game Einstellungen oder Theme Einstellungen für einen selbst.

\end{document}

